Humans can use different forms of communications such as speech, hand gestures and emotions. Being able to understand one’s emotions and the encoded feelings is an important factor for an appropriate and correct understanding. As such systems that can recognize them, allowing for a more diverse and natural way of communication, are in great demand in many fields. It could for example help during counselling and other healthcare related fields. Other fields like surveillance or driver safety could also profit from it. Being able to detect the mood of the driver could help to detect the level of attention, so that automatic systems can adapt. There are many emotions that can be shown on human faces, but most researchers aim to identify six basic emotions, identified by Paul Ekman - anger, disgust. fear, happiness, sadness and surprise.\\

Many methods rely on extraction of the facial region. This can be realized in two ways - through manual inference or an automatic detection approach. Methods often involve the Facial Action Coding System which describes the facial expression deconstructing it into the specific action units (AU). An Action Unit is a facial action like ”raising the InnerBrow”. Multiple activation of AUs can describe the facial expression. Being able to correctly detect AUs is a helpful step, since it allows making a statement about the activation level of the corresponding emotion, but detecting handcrafted facial landmarks can be hard, as the distance between them differs depending on the person. Also, it is significantly harder to determine the facial features of a person when only part of their face is visible or if the lighting conditions are poor.\\

Our approach uses Convolutional Neural Networks, which is a special kind of ANN and have been shown to work well as feature extractor when using images as input and are real-time capable. This allows for the usage of the raw input images without any pre- or post- processing.\\

Whole project is inspired by
"Accelerating Very Deep ConvolutionalNetworks for Classification and Detection"
~\cite{vdcnflsir}
and
"DeXpression: Deep Convolutional Neural" articules ~\cite{dcnnfer}.