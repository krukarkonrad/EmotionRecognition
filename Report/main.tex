\documentclass{article}

\usepackage[a4paper,width=150mm,top=25mm,bottom=25mm]{geometry}
\usepackage[utf8]{inputenc}
\usepackage{natbib}
\usepackage{graphicx}

\title{
    {\includegraphics{images.jpg}}\\
    {Artificial Intelligence}\\
    {\large Emotion recognition with Convolutional neural networks}\\
     \small Leaded by M.Eng Kazimierz Kiełkowicz
    }
    \author{Wojciech Chmura, Konrad Krukar}
    \date{May 2020}



\begin{document}
    \maketitle
    \newpage
    \tableofcontents
    \newpage
    
    \section{Overview}
    This project will show the process of building a Convolutional  neural network to recognize emotions on humans faces.
    This type of application can be use full and be applied  in various system such as security cameras. \\ 
    As we know nonverbal signals sometimes say more then words.
    
    \section{Introduction}
    
    \section{Convolutional Neural Networks}
    
    \section{Dataset description}
    Data that we got is in one *.csv file with two columns: \\
    \begin{itemize}
      \item number of emotion \\
        in rrage 0 - 6:
        \begin{itemize}
            \item 0 - angry
            \item 1 - disgust
            \item 2 - fear
            \item 3 - happy
            \item 4 - sad
            \item 5 - surprise
            \item 6 - neutral
        \end{itemize}
      \item string of pixels \\
        Pixels are in one long string (2304 of them), they are in gray scale 0 - 255.
        We need to preproces them into array of 48x48x1 that will could be reorganised as an picture with 
    \end{itemize}
    
    \section{Model builling}
    
    \section{Implementation}
    
    \section{Quantization}
    
    \section{Results}
    
    \section{Conclusion}

\end{document}
